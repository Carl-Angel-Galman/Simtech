Für Aufgabe 2 wurde das nichtlineare Einspurmodell in Simulink als separates Subsystem umgesetzt. 
Die Umrechnung der Winkelgr\"o\ss en von Grad in Radiant sowie von Radiant in Grad erfolgte im Hauptsystem; 
die umgerechneten Signale wurden anschlie\ss end an das nichtlineare Subsystem \"ubergegeben. 
Innerhalb dieses Subsystems wurden die Modellgleichungen in vier weiteren Untersystemen jeweils einzeln modelliert. 
Diese Untersysteme wurden danach entsprechend der Signalfl\"usse miteinander verbunden, sodass aus den einzelnen Teilgleichungen das vollst\"andige nichtlineare Gesamtmodell entstand.

\begin{figure}[h]
    \centering
    \includegraphics[width=0.95\linewidth]{pictures/Aufgabe2Scope.png}
    \caption{Sprungantwort des nichtlinearen Einspurmodells bei einem Lenkwinkelsprung auf $5^\circ$:
    Schwimmwinkel $\beta$ (in Grad) und Giergeschwindigkeit $\dot{\Psi}$ (in Grad/s).}
    \label{fig:aufgabe2scope}
\end{figure}

Abbildung~\ref{fig:aufgabe2scope} zeigt die Reaktion des nichtlinearen Modells auf einen Lenkwinkelsprung von $0^\circ$ auf $5^\circ$ (bei $t \approx 1\,\mathrm{s}$).
Unmittelbar nach dem Sprung steigt die Giergeschwindigkeit $\dot{\Psi}$ stark an und zeigt ein deutliches \"Uberschwingen mit anschlie\ss end ged\"ampften Schwingungen, bevor sich ein station\"arer Endwert einstellt. 
Der Schwimmwinkel $\beta$ reagiert ebenfalls transient und schwingt kurzzeitig, bleibt jedoch im Vergleich zur Giergeschwindigkeit betragsm\"a\ss ig klein und konvergiert nach der Einschwingphase zu einem nahezu konstanten Wert.
Das ged\"ampfte \"Uberschwingen deutet auf ein stabiles, aber unterd\"ampftes Systemverhalten hin; die station\"aren Endwerte entsprechen dem station\"aren Kurvenfahrtzustand bei konstantem Lenkwinkel.