% ============================================
% Linearisierung Einspurmodell – Herleitung
% (jede Gleichung einzeln, ohne "alles ineinander")
% ============================================

\subsection{Betriebspunkt und Linearisierungsannahmen}
Betriebspunkt (Geradeausfahrt):
\begin{equation}
\delta_0 = 0,\qquad \beta_0 = 0,\qquad \dot{\Psi}_0 = 0.
\end{equation}

Für kleine Winkel um $0$ (in rad) werden die nichtlinearen Terme durch Taylor-Approximationen 1. Ordnung ersetzt:
\begin{align}
\tan(\beta) &\approx \beta, \label{eq:tanlin}\\
\cos(\beta) &\approx 1 \quad \Rightarrow \quad \frac{1}{\cos(\beta)} \approx 1, \label{eq:cosbetalin}\\
\cos(\delta) &\approx 1. \label{eq:cosdeltalin}
\end{align}

% ------------------------------------------------
\subsection{Linearisierung der Schräglaufwinkel}

\subsubsection{Vorderachse}
Aus der Aufgabenstellung:
\begin{equation}
\alpha_v \;=\; \delta \;-\;\frac{l_v\,\dot{\Psi}}{v\,\cos(\beta)}\;+\;\tan(\beta).
\end{equation}
Linearisierung 1. Ordnung mittels \eqref{eq:tanlin} und \eqref{eq:cosbetalin}:
\begin{align}
\alpha_v
&\approx \delta \;-\;\frac{l_v\,\dot{\Psi}}{v}\;+\;\beta.
\end{align}
Damit ergibt sich der linearisierte Schräglaufwinkel an der Vorderachse:
\begin{equation}
\boxed{\alpha_v^{\mathrm{lin}} \;=\; \delta \;+\;\beta \;-\;\frac{l_v}{v}\,\dot{\Psi}}
\end{equation}

\subsubsection{Hinterachse}
Aus der Aufgabenstellung:
\begin{equation}
\alpha_h \;=\; \frac{l_h\,\dot{\Psi}}{v\,\cos(\beta)}\;-\;\tan(\beta).
\end{equation}
Linearisierung 1. Ordnung mittels \eqref{eq:tanlin} und \eqref{eq:cosbetalin}:
\begin{align}
\alpha_h
&\approx \frac{l_h\,\dot{\Psi}}{v}\;-\;\beta.
\end{align}
Damit ergibt sich der linearisierte Schräglaufwinkel an der Hinterachse:
\begin{equation}
\boxed{\alpha_h^{\mathrm{lin}} \;=\; -\beta \;+\;\frac{l_h}{v}\,\dot{\Psi}}
\end{equation}

% ------------------------------------------------
\subsection{Linearisierung der \texorpdfstring{$\dot{\beta}$}{beta-dot}-Gleichung}

Aus der Aufgabenstellung:
\begin{equation}
\dot{\beta} \;=\; -\dot{\Psi} \;+\; \frac{1}{m\,v}
\left(
c_v\,\alpha_v\,\frac{\cos(\delta)}{\cos(\beta)} \;+\;
c_h\,\alpha_h\,\frac{1}{\cos(\beta)}
\right).
\end{equation}

Linearisierung der rein trigonometrischen Faktoren mittels \eqref{eq:cosbetalin} und \eqref{eq:cosdeltalin}:
\begin{equation}
\boxed{
\dot{\beta}^{\mathrm{lin}}
\;=\;
-\dot{\Psi}
\;+\;
\frac{1}{m\,v}
\left(
c_v\,\alpha_v^{\mathrm{lin}}
+
c_h\,\alpha_h^{\mathrm{lin}}
\right)
}
\end{equation}

% ------------------------------------------------
\subsection{Linearisierung der \texorpdfstring{$\ddot{\Psi}$}{Psi-ddot}-Gleichung}

Aus der Aufgabenstellung:
\begin{equation}
\ddot{\Psi} \;=\; \frac{1}{J}
\left(
l_v\,c_v\,\alpha_v\,\cos(\delta)
\;-\;
l_h\,c_h\,\alpha_h
\right).
\end{equation}

Linearisierung des Faktors $\cos(\delta)$ mittels \eqref{eq:cosdeltalin}:
\begin{equation}
\boxed{
\ddot{\Psi}^{\mathrm{lin}}
\;=\;
\frac{1}{J}
\left(
l_v\,c_v\,\alpha_v^{\mathrm{lin}}
\;-\;
l_h\,c_h\,\alpha_h^{\mathrm{lin}}
\right)
}
\end{equation}